See \autoref{lst:toph} for the implementation; here, we will work through the details to the code. \texttt{toph.jl}
is a direct translation
into Julia of the \texttt{toph} listing from the appendices in \cite{bendsoe_sigmund_topopt}. Due to Julia's
design and syntax, the codes end up being quite similar, though the Julia version may be more
familiar to a generally experienced programmer.

The problem is to minimize the thermal compliance in the unit square as was done in \autoref{sec:thermal_compliance_classic_unit_square}.
We are interested in the distribution of two material phases with isotropic conductivities of $1$ and $0.001$ for material and void, respectively.
%\begin{equation}\label{eq:toph_problem}
    %\begin{aligned}
        %\min_{\mathbf{x}} &\quad c(\mathbf{x}) := U^T K U = \sum E_e(x_e) u_e^T k_0 u_e\\
        %\text{s.t.} &\quad V(x)/ V_0 = \theta,\\
            %&\quad KU = F,\\
            %&\quad 0 \leq \rho \leq 1.
    %\end{aligned}
%\end{equation}




\subsection{Primary Loop}
The main optimization loop is as follows:
\begin{algorithm}
    \caption{\texttt{toph}: Main loop.}\label{alg:toph_main}
    \begin{algorithmic}
    \Require \texttt{nelx} (number of $x$ elements), \texttt{nely}, $\theta$ (volume fraction), $p$ (penalization factor), $r_{\text{min}}$ (filter radius)
    \Ensure $x$ satisfies the optimization problem.

    \State $x \gets \theta \cdot \mathbf{1}$ \Comment{Initial guess is a matrix of $\theta$'s}

    \While{not converged}

        $x_\text{old} \gets x$

        $U \gets$ \texttt{FE}; Displacement vector. \Comment{State solution.}

        $KE \gets$ Element Stiffness Matrix

        $c \gets$ objective function calculation

        $dc \gets$ \texttt{check} \Comment{Sensitivity filter.}

        $x \gets$ \texttt{OC} \Comment{Design update step (by the optimality criteria method).}

    \EndWhile
    \end{algorithmic}
\end{algorithm}

The convergence condition depends on the largest value in the difference $x - x_\text{old}$ or the maximum number of allowed iterations.
Here, the only calculation that requires motivation is that of the objective function (the large \texttt{for} loop).
We have seen that this is $U^T K U = \sum E_e(x_e) u_e^T k_0 u_e$. Due to the symmetry between all the square elements, for solid material,
$k_0$ (the element stiffness matrix) is identical,\footnote{The MATLAB code keeps this as a subroutine that returns a constant value,
but this is inefficient in general for several reasons, especially for a JIT language like Julia.} so we need only construct it once and can maintain it as e.g. a \texttt{const}.
\begin{equation}
    KE = \begin{pmatrix}
        k_{11} & k_{12} & k_{13} & k_{14} \\
        k_{21} & k_{22} & k_{23} & k_{24} \\
        k_{31} & k_{32} & k_{33} & k_{34} \\
        k_{41} & k_{42} & k_{43} & k_{44} \\
    \end{pmatrix} = \begin{pmatrix}
        2/3 & -1/6 & -1/3 & -1/6 \\
        -1/6 & 2/3 & -1/6 & -1/6 \\
        -1/3 & -1/6 & 2/3 & -1/6 \\
        -1/6 & -1/3 & -1/6 & 2/3 \\
    \end{pmatrix}.
\end{equation}
We calculate also the sensitivites $\frac{\partial c}{\partial x_e}$, the formula for which we have essentially seen; in particular,
notice that the derivatives are always negative, meaning that increasing the material reduces the thermal compliance of the structure.
\begin{equation}
    \partial c / \partial x_e = -0.999 \cdot p \cdot x_e^{p-1} u^T k_0 u
\end{equation}
\texttt{Ue} denotes the element displacement; obtaining this is a matter of vertex indexation once we have the global displacement vector
\texttt{U}. The elements are numbered column-wise from left to right. This constructed sensitivity matrix is passed into the sensitivity filter subroutine as well as the relevant parameters and current
material distribution.

\subsection{Finite Element Analysis}

We form the global stiffness matrix \texttt{K} with a loop over all elements. Then, the load is set to all $0.01$, meaning that entire domain is
uniformly heated. Finally, the matrix problem $Ku = f$ is solved and the solution $U$ is returned.

\subsection{OC based optimizer}

Here, we follow the algorithm outlined in the discussion of \texttt{top88}; nothing changes in this case with the physics hot-swap. The primary
loop is the bisectioning algorithm for finding the Lagrange multiplier, and while we wait for the bisectioning algorithm, at each step, we update
the distribution in \texttt{xnew}, which is finally returned upon convergence.

\subsection{Mesh-independency filtering}

This is an implementation of the sensitivity filter, which was skipped for the sake of brevity. The scheme modifies the element sensitivities
of the compliance with\cite{bendsoe_sigmund_topopt}
\begin{equation}
    \hat{\frac{\partial f}{\partial \rho_k}} = \frac{1}{\rho_k \sum \hat{H}_i} \sum \hat{H}_i \rho_i \frac{f}{\rho_i},
\end{equation}
where the weight factor $\hat{H}_i$ is
\begin{equation}
    \hat{H}_i = r_\text{min} - \text{dist}(k,i), \quad \{i \in N | \text{dist}(k,i) \leq r_\text{min}\}, k = 1, \ldots, N,
\end{equation}
where $N$ is the number of elements in the mesh.




